\chapter{Tecnologias e Ferramentas Utilizadas}
\label{chap:tecno-ferra}

\section{Introdução}
\label{chap3:sec:intro}
Para a realização deste trabalho foram necessárias diversas ferramentas,
das quais o \textit{Linux}, sendo o sistema operativo onde roda o executável a partir do \textit{Makefile}, \textit{\ac{VS Code}} para a realização do código em linguagem C\cite{devdocsDevDocsx2014C}, o \textit{Doxygen} para gerar automaticamente documentação relativa ao código, o \textit{Overleaf} que, através do \LaTeX, se fez o relatório.

\section{Ferramentas}
\label{chap3:sec:ferramentas}

\subsection{\textit{Linux}}
\label{chap3:subsec:Linux}
O \textit{Linux} é um sistema operativo de código aberto baseado no núcleo \textit{Linux}, desenvolvido por Linus Torvalds em 1991. Ele é amplamente utilizado em servidores, dispositivos móveis e computadores pessoais devido à sua estabilidade, segurança e flexibilidade. O \textit{Linux} é gratuito e permite modificações, o que o torna uma escolha popular para desenvolvedores e empresas. A comunidade por trás do \textit{Linux} garante um ecossistema rico em ferramentas e recursos, além de um suporte técnico robusto.

\newpage

\subsection{Terminal}
\label{chap3:subsec:terminal}
O Terminal é uma interface de utilizador baseada em texto, proporcionando ao utilizador um canal direto para interagir com o sistema operativo subjacente. Dispensando a necessidade de interfaces gráficas, o Terminal opera através da interpretação de comandos de texto, estabelecendo uma comunicação precisa e eficiente entre o utilizador e o sistema.

\subsection{\LaTeX}
\label{chap3:subsec:latex}
O \gls{LateX} é um sistema de preparação de documentos desenvolvido na década de 1980 por Leslie Lamport. Ao escrever, o escritor usa texto simples, ao invés do texto formatado encontrado em processadores de texto como \textit{Microsoft Word} e \textit{LibreOffice Writer}. O escritor utiliza padrões de marcação para definir a estrutura desejada do documento, formatando assim o texto em todo um documento e incluindo citações e referências de forma simples.


\subsection{\textit{Overleaf}}
\label{chap3:subsec:overleaf}
O Overleaf é um editor de LaTeX online que facilita a edição de documentos em LaTeX, permitindo a colaboração em tempo real e tornando mais ágil a criação de textos como relatórios técnicos. Além disso também permite a integração com ferramentas de gestão bibliográfica, tais como \textit{Zotero}, \textit{JabRef}, \textit{Mendeley} e \textit{EndNote}, facilitando as citações e referências de forma prática e organizada.


\subsection{\textit{\acs*{VS Code}}}
\label{chap3:subsec:vscode}
O \gls{VS Code} é um editor de código leve e gratuito, criado pela Microsoft, que se tornou bastante popular entre desenvolvedores. Ele suporta diversas linguagens de programação, como \textit{JavaScript}, \textit{Python} e C++\cite{devdocsDevDocsx2014CPP}, e oferece uma vasta coleção de extensões que permitem personalizar a experiência de codificação. Com ferramentas integradas, como depuração, controle de versão \textit{Git}\cite{gitscm} e suporte a ambientes de desenvolvimento remoto, o \textit{\ac{VS Code}} é uma escolha prática e eficiente para quem busca agilidade e flexibilidade no desenvolvimento de \textit{software}.

\newpage

\subsection{\textit{Make}}
\label{chap3:subsec:make}
O \textit{Make} é uma ferramenta fundamental em ambientes de desenvolvimento, especialmente em projetos que envolvem compilação de código-fonte. Este automatiza o processo de construção de \textit{software}, tornando-o mais eficiente e menos propenso a erros.Através da definição de um \textit{Makefile}, o \textit{Make} permite especificar as dependências entre arquivos e as ações necessárias para compilar e gerar o \textit{software}, garantindo que apenas os componentes alterados sejam recompilados. A utilização do \textit{Make} melhora a organização do código, promovendo uma estrutura mais clara.

\subsection{\textit{Doxygen}}
\label{chap3:subsec:doxygen}
O \textit{Doxygen} é uma ferramenta poderosa e versátil utilizada para gerar documentação de código-fonte automaticamente. Este analisa o código-fonte e, a partir de comentários especiais inseridos, cria uma documentação completa e profissional, facilitando a compreensão e manutenção do seu projeto. Além disso, suporta várias linguagens de programação, como C, C++ e \textit{Python}, e permite gerar documentação em diferentes formatos, como HTML, \LaTeX e PDF.


\section{Conclusões}
\label{chap3:sec:concs}
Em síntese, as ferramentas apresentadas neste capítulo foram essenciais para o desenvolvimento do trabalho. O \textit{Linux} e o Terminal forneceram um ambiente sólido para desenvolvimento, enquanto o \textit{\ac{VS Code}} facilitou a programação em C. O \textit{Make} automatizou a compilação, e o \textit{Doxygen} gerou documentação. Para o relatório, o \LaTeX e o \textit{Overleaf} permitiram a criação de um documento técnico bem estruturado e eficiente. A integração dessas tecnologias garantiu um bom fluxo de trabalho.