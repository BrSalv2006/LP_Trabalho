\chapter{Implementação e Testes}
\label{chap:imp-test}

\section{Introdução}
\label{chap4:sec:intro}

\par Durante a fase de implementação e de testes do código, foi utilizado o \textit{\ac{VS Code}} juntamente com o \textit{Make} para a simplificação e aceleração do processo de compilação de código.
\par
O projeto foi dividido em três ficheiros: \texttt{main-team-8.c}, \texttt{functions-team-8.c} e \texttt{functions-team-8.h}. Na secção seguinte são apresentados alguns excertos destes mesmos ficheiros.

\newpage

\section{Trechos de Código}
\label{chap4:sec:code}

\begin{lstlisting}[caption=Mostrar Menu e Verificar Opção escolhida., label=chap4:code:1]
void Menu(int *OriginalArray) {
    int MenuChoice, ValidMenuChoice;
    PrintIntegerArray(OriginalArray, "Vetor Original:", ArraySize);
    printf("Menu:\n");
    printf("1  - Vetor ordenado por ordem crescente\n");
        %*\texttt{\vdots}*)
    printf("12 - Sair\n");

    // Guardar se escolha é válida
    ValidMenuChoice = scanf("%d", &MenuChoice); 
    ClearTerminal();
    // Verificar se existe uma escolha válida
    if (ValidMenuChoice == 1) {
        switch (MenuChoice) {
        case 1:
            %*\textnormal{\vdots}*)
        default:
            printf("Opção Inexistente\n\n");
            // Abrir Menu
            Menu(OriginalArray);
        }
    } else {
        // Limpar Input
        ClearInput();
        // Abrir Menu
        Menu(OriginalArray);
    }
}
\end{lstlisting}

\newpage

\begin{lstlisting}[caption=Verificação dos argumentos e preenchimento do Vetor., label=chap4:code:2]
int main(int argc, char **argv) {
    // Iniciar Geração Aleatória com SEED baseada no tempo
    srand(time(NULL));
    for (int i = 0; i < argc; i++) {
        // Verificar se o argumento introduzido é "--help"
        if (strcmp(argv[i], "--help") == 0) {
            // Mostrar Ajuda
            Help(ArraySize, Minimum, Maximum);
            printf("Pressione Enter para fechar a Ajuda\n");
            // Esperar por Input do utilizador para fechar a ajuda
            getchar();
            ClearTerminal();
            // Terminar Programa com Sucesso
            return 0;
        }
    }
    // Alocar Memória para o Array
    OriginalArray = AllocateIntegerArray(OriginalArray, ArraySize);
    // Preencher o Array
    RequestArray(OriginalArray, ArraySize, Minimum, Maximum);
    // Abrir Menu
    Menu(OriginalArray);
}    
\end{lstlisting}

\newpage

\begin{lstlisting}[caption={Pedir, Verificar e Guardar Números Introduzidos.}, label=chap4:code:3]
int *RequestArray(int *Array, int ArraySize, int Minimum, int Maximum) {
    int Number, ValidNumber;
    ClearTerminal();
    for (int i = 0; i < ArraySize;) {
        printf("Insira um número inteiro entre %d e %d (%d%*\textnormal{\textordmasculine{}}*) Número): ", Minimum, Maximum, i + 1);
        // Guardar número de caráteres númericos válidos introduzidos
        ValidNumber = scanf("%d", &Number);
        ClearTerminal();
        // Verificar se número introduzido é um caráter não númerico
        if (ValidNumber == 1) {
            // Verificar se número introduzido está entre um Mínimo e um Máximo
            if (Number >= Minimum && Number <= Maximum) {
                // Guardar número no Array
                Array[i] = Number;
                i++;
            } else {
                printf("O número deve estar compreendido entre %d e %d.\n", Minimum, Maximum);
            }
        } else {
            // Limpar Input em caso do caráter introduzido não ser um número
            ClearInput();
        }
    }
    ClearTerminal();
    return Array;
}
\end{lstlisting}

\begin{lstlisting}[caption=Alocar Memória para um Vetor de Inteiros., label=chap4:code:4]
int *AllocateIntegerArray(int *IntegerArray, int ArraySize) {
    IntegerArray = (int *)calloc(ArraySize, sizeof(int)); // Alocar Memória para um Array de Inteiros
    return IntegerArray;
}
\end{lstlisting}

\begin{lstlisting}[caption=Mostrar Vetor, label=chap4:code:5]
void PrintIntegerArray(int *Array, char *Text, int ArraySize) {
    printf("%s\n", Text);
    for (int i = 0; i < ArraySize; i++) {
        printf(" %d", Array[i]);
    }
    printf("\n\n");
}
\end{lstlisting}

\begin{lstlisting}[caption=Ordenar Vetor por Ordem Crescente, label=chap4:code:6]
int *SortAscendingOrder(int *Array, int *IntegerArray, int ArraySize) {
    // Alocar Memória para o Array
    IntegerArray = CopyArray(Array, AllocateIntegerArray(IntegerArray, ArraySize), ArraySize);
    int k;
    do {
        k = 0;
        for (int i = 0; i < ArraySize - 1; i++) {
            // Verificar se o número i é maior que o número i+1
            if (IntegerArray[i] > IntegerArray[i + 1]) {
                // Guardar número i na variável temporária
                int Temp = IntegerArray[i];
                // Colocar número i+1 na posição do número i
                IntegerArray[i] = IntegerArray[i + 1];
                // Colocar número na variável temporária na posição i+1
                IntegerArray[i + 1] = Temp;
                k++;
            }
        }
    } while (k != 0); // Repetir enquanto não ordenado
    return IntegerArray;
}
\end{lstlisting}

\begin{lstlisting}[caption=Adicionar Primeira metade à Segunda Metade do Vetor., label=chap4:code:7]
int *AddFirstSecondHalf(int *Array, int *IntegerArray, int ArraySize) {
    // Alocar Memória para o Array
    IntegerArray = AllocateIntegerArray(IntegerArray, ArraySize / 2);
    for (int i = 0; i < ArraySize / 2; i++) {
        // Somar número i com número i+ArraySize/2
        IntegerArray[i] = Array[i] + Array[i + ArraySize / 2]; 
    }
    return IntegerArray;
}
\end{lstlisting}

\newpage

\begin{lstlisting}[caption=Gerar Matriz de Permutações., label=chap4:code:8]
int *ShuffleArray(int *Array, int ArraySize) {
    for (int i = ArraySize - 1; i > 0; i--) {
        // Gerar posição aleatória para a qual o número vai ocupar
        int j = rand() % ArraySize;
        // Guardar número na posição gerada aleatóriamente na variável temporária
        int Temp = Array[j];
        // Colocar número i na posição do número j
        Array[j] = Array[i];
        // Colocar número na variável temporária na posição i
        Array[i] = Temp;
    }
    return Array;
}

int **GeneratePermutedMatrix(int *Array, int **Matrix, int ArraySize) {
    Matrix = AllocateMatrix(Matrix, ArraySize);             // Alocar Memória para o Array Bidimensional
    CopyArray(Array, Matrix[0], ArraySize);                 // Copiar Array para a primeira linha do Array Bidimensional
    for (int i = 1; i < ArraySize; i++) {                   // Repetir para o Array todo excluindo a primeira linha
        Matrix[i] = CopyArray(Array, Matrix[i], ArraySize); // Copiar Array para a linha i do Array Bidimensional
        ShuffleArray(Matrix[i], ArraySize);                 // Baralhar Linha i do Array
    }
    return Matrix;
}
\end{lstlisting}

Vale realçar que a geração aleatório de números (rand()), usada no Excerto de código \ref{chap4:code:8}, sendo esta baseada no tempo, a mesma gera um número igual se esta for executada no mesmo segundo.

\begin{lstlisting}[caption= Posições Múltiplas de três., label=chap4:code:9]
int *PositionsMultipleof3(int *Array, int *IntegerArray, int ArraySize) {
    // Alocar Memória para o Array
    IntegerArray = AllocateIntegerArray(IntegerArray, ArraySize / 3);
    for (int i = 0; i * 3 + 2 < ArraySize; i++) {
        // Guardar os números em posições múltiplas de 3 num Array
        IntegerArray[i] = Array[i * 3 + 2];
    }
    return IntegerArray;
}
\end{lstlisting}

Ao escrever a função demonstrada no Excerto de código \ref{chap4:code:9} assumimos que o primeiro elemento do Vetor ocupa a posição um.

\begin{lstlisting}[caption=Produto entre o vetor Introduzido e um vetor aleatório., label=chap4:code:10]
int **ProductBetweenTwoArrays(int *Array, int *IntegerArray, int **Matrix, int ArraySize, int Minimum, int Maximum) {
    // Alocar Memória para o Array
    IntegerArray = AllocateIntegerArray(IntegerArray, ArraySize);
    // Alocar Memória para o Array Bidimensiona
    Matrix = AllocateMatrix(Matrix, ArraySize);
    for (int i = 0; i < ArraySize; i++) {
        // Gerar número Aleatório entre dois número e guardar num Array
        IntegerArray[i] = rand() % (Maximum + 1 - Minimum) + Minimum; 
	}
    for (int i = 0; i < ArraySize; i++) {
	   for (int j = 0; j < ArraySize; j++) {
            // Calcular produto entre os dois Arrays e Guardar no Array Bidimensional
            Matrix[i][j] = IntegerArray[i] * Array[j];
        }
    }
    DisposeIntegerArray(IntegerArray); // Libertar Memória do Array
    return Matrix;
}
\end{lstlisting}

\section{Conclusões}
\label{chap4:sec:concs}
Os exemplos de código apresentados mostram que é importante saber programar em C, mas também escrever de forma clara e organizada. Usar comentários no código ajuda muito, porque facilita a leitura e o entendimento por outras pessoas. Além disso, trocar ideias e trabalhar em equipa é essencial, já que há muitas maneiras diferentes de resolver os mesmos problemas em programação.