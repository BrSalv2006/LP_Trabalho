\chapter{Conclusões e Trabalho Futuro}
\label{chap:conc-trab-futuro}

\section{Conclusões Principais}
\label{sec:conc-princ}

Esta secção aborda as principais conclusões obtidas ao longo do desenvolvimento do trabalho.\newline
\textbf{\emph{Quais foram as conclusões principais alcançadas pelo(a) aluno(a) ao final deste trabalho?}}

Com os aprendizados e as informações adquiridas durante o semestre na unidade curricular de \ac{LP}, foi possível realizar este trabalho com relativa facilidade. O conhecimento consolidado ao longo das aulas e práticas foi fundamental para alcançar os objetivos propostos, permitindo o desenvolvimento eficiente e estruturado das atividades.

\section{Trabalho Futuro}
\label{sec:trab-futuro}

Esta secção responde às seguintes questões:\newline
\textbf{\emph{O que ficou por fazer e por quê?}}

Felizmente, todos os objetivos propostos para o programa foram concluídos com êxito. Não restaram pendências ou funcionalidades incompletas, e os erros encontrados durante o desenvolvimento foram devidamente corrigidos. Assim, o programa foi finalizado e está funcional, sem quaisquer impedimentos para sua execução.

\newpage
\textbf{\emph{O que seria interessante fazer, mas não foi implementado por não ser o objetivo principal deste trabalho?}}

No contexto deste trabalho, todos os conteúdos lecionados na unidade curricular foram devidamente incorporados, não deixando lacunas para melhorias adicionais. Contudo, há sempre oportunidades para explorar funcionalidades extras ou aprimoramentos que poderiam ser interessantes, mas que excederiam os objetivos inicialmente definidos.

\textbf{\emph{Em que outros cenários ou aplicações o trabalho aqui descrito pode ser útil, e por quê?}}

A realização deste trabalho proporcionou um aprofundamento significativo nos conhecimentos sobre elaboração de relatórios no \textit{Overleaf}, uma habilidade que será essencial em futuros projetos académicos e profissionais. Além disso, a experiência prática adquirida no uso da linguagem C se mostrou indispensável para a resolução de problemas e desenvolvimento de códigos robustos, abrindo caminhos para aplicações em projetos futuros e em diferentes contextos relacionados ao desenvolvimento de \textit{software}.