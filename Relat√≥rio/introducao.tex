\chapter{Introdução}
\label{chap:intro}

\section{Enquadramento}
\label{sec:amb} 

Este relatório foi desenvolvido no âmbito da unidade curricular de Laboratórios de Programação do curso Licenciatura em Engenharia Informática da \ac{UBI} e tem como finalidade implementar  um programa que peça ao utilizador 20 números inteiros e os guarde num vetor, para  posteriormente  providenciar forma de calcular algumas estatísticas ou fazer operações sobre esses valores. O projeto enquadra-se também na Álgebra e no Cálculo. Este também é importante porque ajuda a reforçar conhecimentos técnicos e a desenvolver habilidades para resolver problemas, o que é essencial para criar soluções práticas na área da tecnologia. \par


\section{Motivação}
\label{sec:mot}
Abordar o problema neste projeto é essencial, ao permitir aplicar de forma prática os conceitos de programação aprendidos, integrando-os na criação de um código de maior complexidade que simula desafios reais da Engenharia Informática. Além disso, o trabalho em grupo promove o desenvolvimento de competências importantes, como organização, colaboração e divisão de tarefas, fundamentais para projetos de larga escala. A utilização do LaTeX para documentar o relatório reforça a capacidade de comunicação técnica, garantindo a clareza e a estruturação necessária para justificar decisões e facilitar a manutenção do código. Assim, o projeto combina teoria, prática e trabalho colaborativo, preparando os alunos para desafios técnicos e profissionais futuros.

\section{Objetivos}
\label{sec:obj}
O objetivo principal deste projeto é desenvolver um programa em C que solicite ao utilizador 20 números inteiros, assegurando que cada valor esteja entre oito e 29, e os armazene num vetor. Posteriormente, o programa deverá apresentar um menu que permita ao utilizador realizar diversas operações e cálculos estatísticos sobre os valores inseridos, como ordenar o vetor, calcular somas específicas, gerar matrizes baseadas no vetor, entre outras funcionalidades. \par
Mas para além deste objetivo ainda existe mais um com uma versão mais elaborada do projeto que implementa funcionalidades adicionais que ampliam as capacidades do programa, tais como: permitir a leitura de um novo vetor e combiná-lo com o anterior, calcular o mínimo múltiplo comum de pares de números consecutivos, gerar e manipular matrizes resultantes do produto de vetores, e fornecer uma página de ajuda acessível tanto pelo menu quanto por meio de uma \textit{flag} na linha de comandos. Estas funcionalidades avançadas têm como propósito enriquecer a experiência do utilizador e oferecer ferramentas mais robustas para a análise dos dados inseridos.


\section{Organização do Documento}
\label{sec:organ}
De modo a refletir o trabalho feito, este documento encontra-se estruturado da seguinte forma:
\begin{enumerate}
\item O primeiro capítulo -- \textbf{Introdução} -- Apresenta o projeto, a motivação para a sua escolha, o enquadramento para o mesmo, os seus objetivos e a respetiva organização do documento.
\item O segundo capítulo -- \textbf{Estado da Arte} -- Aborda as ferramentas fundamentais aplicadas no desenvolvimento deste projeto e analisa outros trabalhos semelhantes com a mesma lógica.
\item O terceiro capítulo -- \textbf{Tecnologias Utilizadas} -- Explica todas as ferramentas usadas ao longo do projeto e os conceitos que precisam de mais detalhes.
\item O quarto capítulo -- \textbf{Implementação e Testes} -- Descreve os passos detalhados na criação do programa, explicando as etapas presentes do trabalho.
\item O quinto capítulo -- \textbf{Conclusões e Trabalho Futuro} -- Apresenta o desfecho do trabalho, analisando os resultados obtidos e sugerindo possíveis melhorias e aplicações futuras.
\end{enumerate}

\section{Delegação de Tarefas}
\label{sec:deleg}

O trabalho foi dividido em tarefas da seguinte forma:
\begin{itemize}
    \item Código: Bruno Salvador;
    \item Documentação (Doxygen): Gabriel Teixeira;
    \item Edição do Relatório: Dinis Miranda e Duarte Rufino;
    \item Revisão do Trabalho: João Gonçalves e João Silva.
\end{itemize}